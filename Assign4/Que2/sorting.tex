\documentclass[a4paper, 10pt,twocolumn]{article}

\usepackage{algorithm}
\usepackage{algpseudocode}

\title{Comparison Based Sorting Algorithms}
\author{Rajnesh Kumar Meena}
\date{}
\begin{document}
\maketitle

\begin{abstract}

This document presents a brief discussion on sorting algorithms.
Algorithms for Quicksort is provided in this document and its working is explained.
Further, a proof of lower bounds on sorting is presented in this document. 
Most of the content presented here is created by referring and reproducing
contents from one of the widely followed book on Algorithms by Cormen et al. [1]
\textbf {We do not claim originality of this work.}
This document is prepared as part of an assignment for the Software Lab
Course (CS251) to learn \LaTeX.
Declaration: I have prepared this document using \LaTeX  without any unfair means. Further,
while the document is prepared by me, I do not claim the ownership of the ideas presented in this document.

\end{abstract} 

\section{Introduction}
Sorting is one of the most fundamental operations in computer science useful in numerous applications. Given a sequence of numbers as input, the
output should provide a non-decreasing sequence of numbers as output. More formally, we define a sorting problem as follows [1],


\noindent\textbf {Input:}A sequence of n numbers $<a_1,a_2,...,a_n>.$

\noindent\textbf {Output:} A reordered sequence (of size n) $ <a_1^{'},a_2^{'}, ...,a_n^{'}> $ of the input sequence such that
$ a_1^{'} \leq a_2^{'}\leq ... \leq a_n^{'}. $

\noindent Consider the following example. Given an input
sequence  $<8,34,7,9,15,91,15>$ , a sorting algorithm
should return $<7,8,9,15,15,34,91>$ as output.

A fundamental problem like sorting has attracted
many researchers who contributed with innovative
algorithms to solve the problem of sorting efficiently [3].
Efficiency of an algorithm depends on primarily on two aspects,
\begin{itemize}
\item \textbf{Time complexity} is a formalism that captures running time of an algorithm in terms of
the input size. Normally, asymptotic behavior
on the input size is used to analyze the time
complexity of algorithms.
\item \textbf{Space complexity} is a formalism that cap-
tures amount of memory used by an algorithm
in terms of input size. Like time complexity
analysis, asymptotic analysis is used for space
complexity.
\end{itemize}

\noindent In the branch of algorithms and complexity in com-
puter science, space complexity takes a back seat
compared to time complexity. Recently, another
parameter of computing i.e., energy consumption
has become popular. Roy et al. [4] proposed an en-
ergy complexity model for algorithms. In this doc-
ument, we will deal with time complexity of sorting
algorithms.

One class of algorithms which are based on element comparison are commonly known
as comparison based sorting algorithms. In this document we
will provide a brief overview of Quicksort, a commonly
used comparison based sorting algorithm [2]. Quicksort
is a sorting algorithm based on divide-and-conquer
paradigm of algorithm design. Further,
we will derive the lower bound of any comparison based
sorting algorithm to be $\Omega(nlog_{2}n)$ for an input size of n.


\section{Quicksort}

Quicksort is designed as a three-step divide-and-
conquer process for sorting an input sequence in
an array [1]. For any given subarray, $A[i..j]$, the
process is as follows,

\textbf{Divide} The array $A[i..j]$ is partitioned into two
subarrays $A[i..k]$ and $A[k + 1..j]$ such that all ele-
ments in $A[i..k]$ is less than or equal to all elements
in $A[k + 1..j]$. A partitioning procedure is called to
determine k such that at the end of partitioning, the element at the $k^{th}$
position (i.e., $A[k]$) does not change its position in the final output array.

\begin{algorithm}
\caption{Partition procedure of Quicksort algorithm.}
\begin{algorithmic}[1]
     \Procedure{Partition}{A,i,j}\newline
     \Comment{A is an array of N integers, A[1..N]} \newline
     \Comment{i and j are the start and end of subarray}	
      \State $x \leftarrow A[i]$
	\State $y \leftarrow i-1$
	\State $z \leftarrow j+1$
	\While {(true)}
	 \State $z \leftarrow z-1$
          \While {$A[z] > x$}
           \State $z \leftarrow z-1$ 
	  \EndWhile 
	  \State $y \leftarrow y+1$
	\While {$A[y] < x$}
	  \State $y \leftarrow y+1$
	\EndWhile
	\If {$y < z$}
	    \State Swap $A[y] \leftrightarrow A[z]$ 
	\Else
 	   \State return $z$ 
	\EndIf
      \EndWhile
     \EndProcedure 
  \end{algorithmic}
\end{algorithm}

\textbf{Conquer:} Recursively invoke Quicksort on the two subarrays. This procedure conquers the
complexity by applying the same operations in two subarrays.

\textbf{Merge:} Quicksort does not require merge or combine operation as the entire array $A[i..j]$ is sorted in place.

In the heart of Quicksort, there is a partition
procedure as shown in Algorithm 1. A pivot element $x$ is selected. The first inner while loop (line
\#6) continues examining elements until it finds an
element that is smaller than or equal to the pivot element. Similarly, the second inner while loop (line
\#9) continues examining elements until it finds an
element that is greater than or equal to the pivot
element. If indices $y$ and $z$ have not exchanged
their side around the pivot, the elements at $A[y]$
and $A[z]$ are exchanged. Otherwise, the procedure
returns the index $z$, such that all elements to the
left of $z$ are smaller than or equal to $A[z]$ and all
elements to the right of z are greater than or equal
to $A[z]$.

\begin{algorithm}
\caption{Quicksort recursion.}
\begin{algorithmic}[1]
     \Procedure{Quicksort}{A,i,j}\newline
     \Comment{Quicksort procedure called with A,1,N} \newline
	\If {$i < j$}
	    \State $k \leftarrow$ Partition(A,i,j)
	    \State Quicksort(A,i,k)
	    \State Quicksort(A,k+1,j)
	\EndIf
     \EndProcedure 
  \end{algorithmic}
\end{algorithm}



The main recursive procedure for Quicksort is presented in Algorithm 2. Initial invocation is performed
by call QUICKSORT($A,1,N$) where N is the length of array N.

\subsection{Time complexity analysis of Quicksort}
The worst case of Quicksort occurs when an ar-
ray of length N , gets partitioned into two subarrays
of size N-1 and 1 in every recursive invocation of
QUICKSORT procedure in Algorithm 2. The par-
titioning procedure presented in Algorithm 1, takes
$\Theta(n)$ time, the recurrence relation for running time
is,

	\[T(n)=T(n - 1)+\Theta(n)\]

As it is evident that $T(1) = \Theta(1)$, the recurrence is solved as follows,

\[T(n)=T(n-1)+\Theta(n)\]
\[=\sum_{k=1}^n \Theta(k)\]
\[=\Theta(\sum_{k=1}^n k)\]
\[=\Theta(n^2)\]

Therefore, if the partitioning is always maximally
unbalanced, the running time is $\Theta(n^{2})$. Intutively,
if an input sequence is almost sorted, Quicksort
will perform poorly. In the best case, partitioning
divides the array into two equal parts. Thus, the
recurrence for the best case is given by,
\[T(n) = 2T(\frac{n}{2}) + \Theta(n)\]
which evaluates to $\Theta(nlog_{2}n)$. Using a compara-
tively involved analysis, the average running time
of Quicksort can be determined to be $O(nlgn)$.


\section{Lower bounds on comparison sorts}
An interesting question about sorting algorithms
based on comparisons is the following: What is
the lower bound of this class of sorting algorithms? This question is important for algorithm
researchers to further improve the sorting algorithms

A decision tree based analysis leads to the following theorem [1].


\noindent \textbf{Theorem 1.} \textsl{Any decision tree that sorts n elements has height $\Omega(nlog_{2}(n)$.}

\textsl{Proof.} Consider a decision tree of height h that
sorts n elements. Since there are n! permutations of n elements, each permutation representing a 
distinct sorted order, the tree must have at least n!leaves. Since a binary tree of height h has no more
than $2^{h}$ leaves. So,

$n!\leq2^{h}$ 


\noindent{Applying logarithmic $(log_{2})$, the inequality becomes,}

$h \geq lg(n!)$.

\noindent{Applying Stirling’s approximations,}

$n! >(\frac{n}{e})^{n}$,

\noindent{where e is natural base of logarithms. Further,}

\[h \geq lg(\frac{n}{e})^{n}\]
\[=nlgn-nlge\]
\[=\Omega(nlgn)\] \

\section{Conclusion}
In this document, we have provided a discussion
on sorting algorithms. We included algorithms for
Quicksort and explained its working. Further, a
proof of lower bounds on sorting is presented in this
document. Most of the content presented here is
created by referring and reproducing contents from
one of the widely followed book on Algorithms by
Cormen et al. [1]. We do not claim originality of
this work. This document is prepared as part of an
assignment for the Software Lab Course (CS251) to
learn \LaTeX.	

\begin{thebibliography}{25}
\bibitem{cormen}
Cormen, T. H., Leiserson, C. E., Rivest,R. L., and Stein, C. \textit{Introduction to Algorithms, Third Edition}
3rd ed. The MIT Press,2009.

\bibitem{Hoare}
Hoare, C. A. R. Algorithm 64: Quicksort. \textit{Communications of} ACM 4, 7 (1961), 321–.

\bibitem{Martin}
Martin, W. A. Sorting. ACM \textit{Computing Survey} 3, 4 (1971), 147–174.

\bibitem{Roy}
Roy, S., Rudra, A., and Verma, A. An energy
complexity model for algorithms
In \textit{Proceedings of the 4th Conference on Innovations in Theoretical Computer Science}
(2013), ITCS'13, pp. 283–304.
\end{thebibliography}
\end{document}
